\chapter{Introduction}  \label{ch:introduction}
In this chapter we give an introduction to the problem addressed in this thesis.


\section{Applications}
Chapters may include sections.

To make sure that this document renders correctly, execute these commands:
\begin{quote}
\begin{verbatim}
pdflatex thesis
bibtex thesis
pdflatex thesis
pdflatex thesis
\end{verbatim}
\end{quote}
Here, the \verb|pdflatex| command may need to be executed three times in order to generate the table of contents and so on. 
Note that a good thesis has figures and tables; examples can be found in Figure~\ref{fig:afigure} and Table~\ref{tab:atable}. And every thesis has references, like~\cite{brilliantgift15}.

\begin{figure}
\begin{center}
\tikzset{
  head/.style = {fill = orange!90!blue,
                 label = center:\textsf{\Large H}},
  tail/.style = {fill = blue!70!yellow, text = black,
                 label = center:\textsf{\Large T}}
}

\begin{tikzpicture}[
    scale = 1.5, transform shape, thick,
    every node/.style = {draw, circle, minimum size = 10mm},
    grow = down,  % alignment of characters
    level 1/.style = {sibling distance=3cm},
    level 2/.style = {sibling distance=4cm}, 
    level 3/.style = {sibling distance=2cm}, 
    level distance = 1.25cm
  ]
  \node[fill = gray!40, shape = rectangle, rounded corners,
    minimum width = 6cm, font = \sffamily] {Coin flipping} 
  child { node[shape = circle split, draw, line width = 1pt,
          minimum size = 10mm, inner sep = 0mm, font = \sffamily\large,
          rotate=30] (Start)
          { \rotatebox{-30}{H} \nodepart{lower} \rotatebox{-30}{T}}
   child {   node [head] (A) {}
     child { node [head] (B) {}}
     child { node [tail] (C) {}}
   }
   child {   node [tail] (D) {}
     child { node [head] (E) {}}
     child { node [tail] (F) {}}
   }
  };

  % Filling the root (Start)
  \begin{scope}[on background layer, rotate=30]
    \fill[head] (Start.base) ([xshift = 0mm]Start.east) arc (0:180:5mm)
      -- cycle;
    \fill[tail] (Start.base) ([xshift = 0pt]Start.west) arc (180:360:5mm)
      -- cycle;
  \end{scope}

  % Labels
  \begin{scope}[nodes = {draw = none}]
    \path (Start) -- (A) node [near start, left]  {$0.5$};
    \path (A)     -- (B) node [near start, left]  {$0.5$};
    \path (A)     -- (C) node [near start, right] {$0.5$};
    \path (Start) -- (D) node [near start, right] {$0.5$};
    \path (D)     -- (E) node [near start, left]  {$0.5$};
    \path (D)     -- (F) node [near start, right] {$0.5$};
    \begin{scope}[nodes = {below = 11pt}]
      \node [name = X] at (B) {$0.25$};
      \node            at (C) {$0.25$};
      \node [name = Y] at (E) {$0.25$};
      \node            at (F) {$0.25$};
    \end{scope}
    \draw[densely dashed, rounded corners, thin]
      (X.south west) rectangle (Y.north east);
  \end{scope}
\end{tikzpicture}
\end{center}
\caption{Every thesis should have figures.\label{fig:afigure}}
\end{figure}

\begin{table}
\begin{center}
\begin{tabular}{ll}
Column A & Column B\\
\hline
Point 1 & Good\\
Point 2 & Bad\\
\end{tabular}
\end{center}
\caption{Every thesis should have tables.\label{tab:atable}}
\end{table}

Final reminder: this template is just an example, if you want you can make adjustments; also discuss with your supervisor which layout he or she likes. But the front page should be as it is now.

TODO: quite a lot!

\section{Thesis Overview}
It is recommended to end the introduction with an overview of the thesis. This chapter contains the introduction; Chapter~\ref{ch:definitions} includes the definitions; Chapter~\ref{ch:relatedwork} discusses related work; Chapter~\ref{ch:evaluation} evaluates the contributions; Chapter~\ref{ch:conclusions} concludes.

Also make a nice sentence with ``bachelor thesis'', LIACS and the names of the supervisors.

